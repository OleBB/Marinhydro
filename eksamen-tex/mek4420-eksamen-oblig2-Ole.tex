\documentclass{article}

\usepackage{simonspreamble}

\usepackage[toc]{appendix}
\usepackage{xcolor}
\usepackage{cancel} % \cancel, \bcancel,  \xcancel  artig: \cancelto
\usepackage{multicol}
\usepackage{blindtext}
\usepackage{lipsum}  %lorem ipsum
\usepackage{comment}  
\usepackage{placeins} %for å trikse med bildeplassering \FloatBarrier

\definecolor{red}{RGB}{255,0,0}
\definecolor{blue}{RGB}{0,0,255}
\definecolor{green}{RGB}{0,128,0}

\renewcommand\thesubsection{\alph{subsection})} %Gjør at subsection blir a) i stedet for 1.1.1
\newcommand{\vect}[1]{\boldsymbol{{#1}}} %en ny kommando som sier at \vect gir fet skrift. det er alternativ vektornotasjon.

\captionsetup[figure]{name=Figur} %på norsk

%\begin{multicols*}{2}
%\end{multicols*}

%--------------

%%%%%% Skriv emnekode her %%%%%%
\makeatletter
\edef\headcourse{MEK4420} %subtitle name and course header
\makeatother

%%%%%% tittel, namn, dato %%%%%%
\title{Krefter og respons i hiv}
\subtitle{\textit{\headcourse}}
\author{\textsc{Ole Sandok}}
\date{March 2025} %\textsuperscript{17}

%%%%%% header title %%%%%%
\makeatletter
\edef\headtitle{}
\makeatother


\begin{document}
\maketitle
\thispagestyle{fancy} 

\section{Introduksjon}
I denne oppgaven ser vi på en flytende geometri i to dimensjoner. Vi regner først på kreftene fra en geometri i bevegelse til tillestående vann. Så regner vi på kreftene som mottas og reflekteres av en stillestående geometri. Vi undersøker resonans og repons i hiv. 

\section{7.1 Randverdiene for problemet vårt}
\subfile{SECTIONS/710BVP/bvp.tex}
%\textcolor{red}{se over og tenke gjennom, er dette korrekt?}

\section{7.2 Randverdiene for Green-funksjonen}
\subfile{SECTIONS/720BVPGREEN/bvpgreen.tex}
\textcolor{red}{gjøre oppgaven}

\section{7.3 Integrallikningen utledet ved Greens teorem}
\subfile{SECTIONS/730GREENINTEGRAL/greenintegral.tex}
\textcolor{red}{gjøre oppgaven 7.3: use greens theorem to derive an integral eq for the heave problem w/ free surface.}
%ferdig
\section{7.4 - Integrallikningen} 
\subfile{SECTIONS/740INTEGRAL/integral.tex}
%ferdig
\section{7.5.1 Diskretisering av svømmeflaten} 
\subfile{SECTIONS/751LAGBOKS/lagboks.tex}
%ferdig 
\section{7.5.2 - Løsning av integrallikningen med kjente variabler}
\subfile{SECTIONS/752SOLVE/solve.tex} 
\textcolor{red}{}
%ferdig
\section{7.5.3 - Løsning av problemet i hiv}
\subfile{SECTIONS/753SOLUTION/753solution.tex}

\section{7.6 - Potensialet $\phi_2$ i fjernfeltet}
\subfile{SECTIONS/760FARFIELD/farfield.tex}
\textcolor{red}{arbeid gjenstår: Vise integralets fra overgang fra (102-104)}
%ferdig
\section{7.7 Utgående bølgeamplitude}
\subfile{SECTIONS/770AMPLITUDENUTOVER/amplitudenutover.tex}
%\textcolor{red}{alternativt-se over hva i denne 7.7 oppgaven som samsvarer med neste 7.8}

\section{7.8 Addert masse}
\subfile{SECTIONS/780ADDEDMASS/addedmass.tex}
%\textcolor{red}{se over på nytt}

\section{7.9 Tilnærmet løsning}
\subfile{SECTIONS/790NEWMAN/newman.tex}
%\textcolor{red}{se over og sammenlikne med 7.11.4?}
\textcolor{red}{sammenlikne fullt utregnet b22 med forenklet b22?}
 
 % - - - - - - - - - - - - - - - - - - - - - - - - - - - - - - - - - %
 
\section{7.10 Diffraksjonsproblemet}
\subfile{SECTIONS/7100DIFFRAKSJON/7100diffraksjon.tex}

\section{7.10.1 Eksitasjonskraft }
\subfile{SECTIONS/7101EXCITING/7101exciting.tex}
%\textcolor{red}{task: obtain numerically the excinting force. Scriptet vårt løser dette og viser graf i neste deloppgave. }

\section{7.10.2 Haskind relations}
\subfile{SECTIONS/7102HASKIND/7102haskind.tex}
\textcolor{red}{Vise at X2/rhog = i A2-infty. Kalkulere ferdig Haskind1 og 2 i MATLAB}

\section{7.11 Body response in heave}
\subfile{SECTIONS/7110BODYRESPONSE/7110bodyresponse.tex}
\textcolor{red}{utlede likningen fra F(t)? legge til mange flere steg fra kladdeboken?  }

\section{7.11.1 Body response in heave}
\subfile{SECTIONS/7111RESONANCE/7111resonance.tex}
%\textcolor{red}{}

\section{7.11.2-3-4 Response as function of frequency}
\subfile{SECTIONS/7112RESPONSEPLOT/7112responseplot.tex}
%\textcolor{red}{sjekk om det er rett b22 }

\section{7.12 Konklusjon }
\subfile{SECTIONS/7120CONCLUSION/7120conclusion.tex}
%\textcolor{red}{task: Write down in 10 main points, A) what you have done, B) what you have found, and C) the main perspectives of this kind of analysis (the main conclusions) regarding the force and response calculations of the heave problem. }

\begin{enumerate}
	\item Hva har blitt gjort: Jeg har skrevet opp en rekke formler     og har forsøkt å vise fremgangsmåten til alle utledningene stegvis. Til tider noe møysommelig.
	\item Jeg har implementert en løser for å beregne potensialene til 4 enkle geometrier. Ved å sette sammen det vi lærer fra Radiasjonsproblemet og Diffraksjonsproblemet har vi kunnet implementere en løser for responsen til vårt objekt i vannet.
	\item 
	\item 
	\item Funnet: Har funnet resonansfrekvensen til de fire geometriene.
	\item Vi funnet ut at det er viktig å korrigere for addert masse.
	\item  
	\item Forståelse: Har tydelig sett at det er stor forskjell på lange, flate geometrier og korte, dype geometrier. Lange reagerer ulikt på forskjellige bølger.
	\item Vi har sett at resonansfrekvensen øker når lengden på geometrien øker.
	\item 
\end{enumerate}


\section{References}
[1]: Marine Hydrodynamics, J. Newman.
[2]: Forelesningsnotater MEK4420, J. Grue. 


\end{document}

