%- solve problem 6.17 in newman for the 2d case obtaining approzimations of the exciting force and damping force. compare the approxiamte solution to the full calculation of b22
%hint: in the calculation of the exciting force in prob 6.17 use of the froud-Krylov approximation is made, obtaining, in the 2d case, 

%%%%%%%%
%Vi ser nå på .. %her er det lov med en språkvask
%- hydrostatisk gjenopprettende kreftene, koeffisientene
%- Antar slank bøye, og sløyfer addert masse og dempning. 
%- Estimer eksitasjonskraften ved Froude-Krylovkraften
%- finn dempningskraften fra Haskindrelasjonen. 
%- beregn respons i hiv, sammenlikn med fig. 2.16.

%%% Mangler Moment
Vi ser nå på en fritt flytende tynn bøye, sylinderformet med diameter $d$ og med dypgang $T$. Mye tilsvarende vår fjerde geometri med $L/D = 0.1$.
%Vi ønsker å finne de hydrostatisk gjenopprettende kreftene og momentene.  Og å estimere den naturlige frekvensen i hiv, med antagelsen om at bøyen er såpass tynn at vi kan sløyfe koeffisientene for addert masse og dempning. Fordi bøyens masse dominerer. 

%steg 1 under.Vi estimerer eksitasjonskraften ved Froude-Krylov approksimasjon. 
%steg 2 og dempningskoeffisienten fra Haskindrelasjonen. 
%steg 3 Vi finner så responsen i hiv. Og sammenlikner med fig.2.16, Newman

%%% FK
Vi starter med utgangspunkt i følgende fra Froude-Krylov approksimasjon for å estimere eksitasjonskraften %FK sin approx?
\begin{equation}
	X_2^{FK} = - \mathrm{i} \omega \rho \int_{S_B} \phi_0 n_2 dS =\rho g \int_{-L/2}^{L/2} e^{-KD- \mathrm{i}Kx} dx = \rho g L e^{-KD} \frac{\sin(KL/2)}{KL/2} 
\end{equation}
der $D$ er dypgang og $L$ er lengden. 
%%%%%%med understrek streket ut phi7
%\begin{equation} X_2^{FK} = - \mathrm{i} \omega \rho \int_{S_B} \underbrace{(\phi_0 + \xcancel{\phi_7})}_{FK \text{ ignorerer }\phi_7} n_2 dS = \rho g \int_{-L/2}^{L/2} e^{-KD- \mathrm{i}Kx} dx = \rho g L e^{-KD} \frac{\sin(KL/2)}{KL/2} \end{equation}
Som gir oss
\begin{equation} X_2^{FK} 
 = \rho g L e^{-KD} , \quad \text{ fordi } \frac{\sin(KL/2)}{KL/2} \simeq 1,  \frac{KL}{2} \ll 1
\end{equation}

vi har 
$c_{22} = \rho g L$.
Fra Haskindrelasjonene finner vi dempningskoeffisienten $b_{22}$,
\begin{align}
	 \frac{b_{22}}{\rho \omega} &= \frac{|X_2|^2 }{(\rho g)^2}\\
	 \frac{b_{22}}{\rho \omega} &= \frac{\cancel{(\rho g)^2} L^2 e^{-2KD}}{\cancel{(\rho g)^2}}.
\end{align}

Vi finner så responsen i hiv. 
\begin{align}
	(c_{22} - &\omega^2(m_{22} + a_{22}) + \mathrm{i}\omega b_{22})\xi_2 = AX_2\\
	 \frac{\xi_{2}}{A} &= \frac{X_2 }{(c_{22} - \omega^2(m_{22} + \xcancel{a_{22}}) + \mathrm{i}\omega b_{22})}\\
	 \frac{\xi_{2}}{A} &= \frac{\bcancel{\rho} g L e^{-KD} }{(\bcancel{\rho} g L - \omega^2(\bcancel{\rho} L D) + \mathrm{i}\omega  (\bcancel{\rho}\omega L^2 e^{-2KD}) }\\
	 \frac{\xi_{2}}{A} &= \frac{ \frac{g}{g} L e^{-KD} }{( \frac{g}{g} L -  \frac{\omega^2}{g}L D + \mathrm{i}\underbrace{(\frac{\omega^2}{g})}_{=k} L^2 e^{-2KD} }\\
	 \frac{\xi_{2}}{A} &= \frac{1 \cancel{L} e^{-KD}}{ 1 \cancel{L} - K \cancel{L} D + \mathrm{i} k {L}^{\cancel{2}} e^{-2KD} }\\
	 \frac{\xi_{2}}{A} &= \frac{ e^{-KD}}{ 1 - K  D + \mathrm{i} K {L} e^{-2KD} }
\end{align}








