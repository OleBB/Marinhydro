% - % - % - % - % - % - % - %  GOLF BUOY % - % - % - % - % - % - % - %
\subsection{Testing the buoy}

\subsubsection{Buoy set up}
The IMU was set to record and write at 100Hz, but the output was only 86Hz. Accelerometer, gyroscope and magnetometer turned was on. Internal digital lowpass filter was turned on. The IMU was connected to the battery and put into the golf buoy. Kitchen paper was used to hold it in place, and to be able to soak up any water that might enter. An attempt at fastening the device to a specific location, with the accelerometer in center, should have been attempted.

\subsubsection{Tank and probe properties}
The wave tank is 24.6m long, 0.50m wide. Water depth is 0.60m. The buoy moved from straight above the mooring point, over to maximum length of the mooring line. Which was for most of the time circa 12m from the wave maker, and 2.5m from the 3rd probe. (My mooring was loose, and did move a few centimeters away from the wave maker during the whole experiment). 
\subsubsection{Computer trigger and selection}
Wave probe data was only 60 seconds or less. This was a mistake, because the first waves took more than that to finish. We then got incomplete data. %Probes should have been triggered with the wave maker.

\subsubsection{Buoy test, November 4th.}
Six waves were run. After this, some extra waves were run that made it hard to distinguish the six planned waves.

\[
\begin{bmatrix}
Run & f [Hz] & Wave maker input [V]  \\ %& Amplitude [cm]
1st   & 0.5 & 1.0 &  \\
2nd & 0.5 & 0.1 &  \\
3rd & 0.5 & 0.3 &  \\
4th & 1 & 0.3 & \\
5th & 2 & 0.3 & \\
6th & 3 & 0.3 &  
\end{bmatrix}
\]

\subsubsection{Data analysis}
Post processing was not trivial. It was not immediately obvious to what data fit together. First, the accelerations as a result of putting the buoy into the tank was much larger than the waves.  And then there was a fair bit of work involved getting the data from the 6 runs to overlay the acceleration data. And because there was no exact knowledge of the time, this had to be adjusted manually. See figure \ref{fig:find}

\begin{figure}[h!]%[H]
    \centering
    \includegraphics[width=0.7\linewidth]{small_wave.jpg}
    \caption{Overlaying the limited probe data, to see what waves we can compare. The pattern is clear, but a slight adjustment of time has to be made. Also, notice a sudden disturbance between waves 2 and 3. This was due to the unexpected arrival of a duck into the tank.} \label{fig:find}
\end{figure}

%%
%Acceleration is not the same as elevation. Even though the acceleration curves looked nicer. %the wave measured at the probe was about 2 meters before the buoy.


%%%%%% OVERVIEW
\begin{figure}[ht] 
    \centering
    \includegraphics[width=1\linewidth]{Log16_500til1477_all_inkl_tidsjusteringer_etadeltpå6.jpg}
    \caption{For illustration purposes, we have an overview of the 6 waves, and corresponding probedata. They are name waves 1 to 6. The FFT data has been amplified so we can compare elevation of the probe measurements. We see that the data seems reasonable, and we can go into a more detailed analysis. However, we see the second wave, probedata shown in green looks strange.}
    \label{fig:}
\end{figure}
%%%%%% CLOSE UP WAVE2
\begin{figure}[h!] 
    \centering
    \includegraphics[width=0.7\linewidth]{Log16_500til1477_all_inkl_tidsjusteringer_etadeltpå6_ allprobes_nærbilde_wave2.jpg}
    \caption{Up-close view of the smallest amplitude wave, (the 2nd wave). Probe 1 magenta, looks fine, probe 2 cyan, looks fine, but probe 3 in green looks bad, as we saw in the overview, and probe 4 in red looks horrible. This is probably due to the other wave buoys disturbing the data }
    \label{fig:}
\end{figure}
%%% EXTRA CLOSEUP WAVE 2
\begin{figure}[h!] 
    \centering
    \includegraphics[width=0.7\linewidth]{Log16_500til1477_all_inkl_tidsjusteringer_etadeltpå6_probe1og2_ekstranærbilde_wave2.jpg}
    \caption{Even closer view. Note: not adjusted to phase or vertical scale. We see the waves measured by both probe 1, magenta, and probe 2, red, measures approximately the same number and same period as the buoy. }
    \label{fig:}
\end{figure}
%Notice also in the middle of the image, there is a slight elevation, that all the

%%%%% - wave data FFT settings visual blue %%%%
\begin{figure}[htbp]
    \centering
    \begin{subfigure}[b]{0.49\textwidth}
        \centering
        \includegraphics[width=\textwidth]{log16_all_fft_f02f1.png}
        \caption{}
        \label{fig:}
    \end{subfigure}
    \hfill
    \begin{subfigure}[b]{0.49\textwidth}
        \centering
        \includegraphics[width=\textwidth]{log16_all_fft_f02f10.png}
        \caption{}
        \label{fig:}
    \end{subfigure}
    \caption{Golf buoy data. Clearly, a change in the post processing, cutoff frequency f2=1 vs f2=10, gives two contradictory results.}
    \label{fig:}
\end{figure}

%%%%%%%%%%% WAVES AND WAVE CLOSEUP
% WAVE 1
\begin{figure}[ht]
    \centering
    \includegraphics[width=0.9\linewidth]{Log16_500til_inkl_tidsjusteringer_etadeltpå9_probe3_wave1.jpg}
    \caption{Wave 1 , probe 3. Y-axis in meters}
    \label{fig:}
\end{figure}
\begin{figure}[ht]
    \centering
    \includegraphics[width=0.9\linewidth]{Log16_500til750_inkl_tidsjusteringer_etadeltpå9_probe3_nærbilde_wave1.jpg}
    \caption{Wave 1 , probe 3. Y-axis in meters}
    \label{fig:}
\end{figure}
%% WAVE 2 
\begin{figure}[ht]
    \centering
    \includegraphics[width=0.9\linewidth]{Log16_1050til1300_inkl_tidsjusteringer_etadeltpå9_probe2_nærbilde_wave2.jpg}
    \caption{Wave 2 , probe 2. Y-axis in meters}
    \label{fig:}
\end{figure}
\begin{figure}[ht]
    \centering
    \includegraphics[width=0.9\linewidth]{Log16_1050til1300_inkl_tidsjusteringer_etadeltpå9_probe2_wave2.jpg}
    \caption{Wave 2 , probe 2. Y-axis in meters}
    \label{fig:}
\end{figure}
%%% WAVE 3
\begin{figure}[ht]
    \centering
    \includegraphics[width=0.9\linewidth]{Log16_1626til1826_inkl_tidsjusteringer_etadeltpå9_probe3_wave3.jpg}
    \caption{Wave , probe 3. Y-axis in meters}
    \label{fig:}
\end{figure}
\begin{figure}[ht]
    \centering
    \includegraphics[width=0.9\linewidth]{Log16_1626til1826_inkl_tidsjusteringer_etadeltpå9_probe3_nærbilde_wave3.jpg}
    \caption{Wave 3, probe 3. Y-axis in meters}
    \label{fig:}
\end{figure}
%%%% WAVE 4
\begin{figure}[ht]
    \centering
    \includegraphics[width=0.9\linewidth]{Log16_1936til2056_inkl_tidsjusteringer_etadeltpå9_probe3_wave4.jpg}
    \caption{Wave 4, probe 3. Y-axis in meters}
    \label{fig:}
\end{figure}
\begin{figure}[ht]
    \centering
    \includegraphics[width=0.9\linewidth]{Log16_1936til2056_inkl_tidsjusteringer_etadeltpå9_probe3_nærbilde_wave4.jpg}
    \caption{Wave 4 , probe 3. Y-axis in meters}
    \label{fig:}
\end{figure}
%%%%% WAVE 5
\begin{figure}[ht]
    \centering
    \includegraphics[width=0.9\linewidth]{Log16_2996til2272_inkl_tidsjusteringer_etadeltpå9_probe2_wave5.jpg}
    \caption{Wave 5, probe 2. Y-axis in meters}
    \label{fig:}
\end{figure}
\begin{figure}[ht]
    \centering
    \includegraphics[width=0.9\linewidth]{Log16_2996til2272_inkl_tidsjusteringer_etadeltpå9_probe2_nærbilde_wave5.jpg}
    \caption{Wave 5, probe 2. Y-axis in meters}
    \label{fig:}
\end{figure}
%%%%% WAVE 6  
\begin{figure}[ht]
    \centering
    \includegraphics[width=0.9\linewidth]{Log16_2452til2470_inkl_tidsjusteringer_etadeltpå9_probe3_nærbilde_wave6.jpg}
    \caption{Wave 6, probe 3. Y-axis in meters}
    \label{fig:}
\end{figure}


\begin{comment}
%PROBEDATA , :  Last excercise we saw that tall 15cm waves loose about X cm per 3.5 m.
d
\end{comment}



