\subsection{Setup}

\subsection{Buoy design}
A well designed buoy should follow the water such that the sensor inside gives meaningful results. As an experiment, the buoy was given a golf ball pattern. And for practical purposes, the "golf bouy" was made to fit a standard IKEA lid.

\subsubsection{3d printing}
The 3d printer (Prusa MK4) with 0.4 nozzle, using PLA, did a fantastic job. No water was found to go through the shell after 4 days under 30 cm of water.  Infill was set to 15\% but, with extra thick walls. Some water was found underneath the flat rubber band of the lid, but this was not a problem.
%0.20 mm speed. generic pla. original prusa MK4 input shaper 0.4 nozzle
%how big was the walls on the onshape fig?. and then x% extra size for shrinkage. 


\begin{figure*}[h!]
    \centering
    \includegraphics[width=0.3\linewidth]{IMG_7293.jpeg}
    \caption{Printing the golf buoy}
    \label{fig:coord}
\end{figure*}

\begin{figure*}[ht]
    \centering
    \includegraphics[width=0.7\linewidth ,angle =-90]{IMG_7549.jpeg}
    \caption{Buoys in the tank. Third and fourth wave probes are seen in the top}
    \label{fig:buoys}
\end{figure*}

\begin{figure*}[h]
    \centering
    \includegraphics[width=0.7\linewidth]{golfbuoy_1225.png}
    \caption{Photo of the golf buoy in a steep wave}
    \label{fig:golfbuoy_1225}
\end{figure*}

\begin{figure*}[h]
    \centering
    \includegraphics[width=0.7\linewidth]{matbuoy_1126}%1126 er tidspunktet bildet er tatt(denne fra 21 okt.
    \caption{This square box was used to study how different designs react to waves. As expected, it rotated unpredictably.}
    \label{fig:matbuoy}
\end{figure*}




% - % - % - % - % - % - % - % - SENSOR VALIDATION % - % - % - % - % - % - % - % -
\subsubsection{Validating the sensor accuracy.}
In order to understand our sensors we did a controlled validation of the IMU. We mounted the accelerometer onto the wavemaker, and made 6 runs with the following frequencies and amplitudes. The 2nd and 6th run were the same.
\[
\begin{bmatrix}
Run & f [Hz] & Wave maker input [V] & Amplitude [cm] \\
1st   & 1 &0.3 & 2.5 \\
2nd & 3 & 0.3 & 2.5 \\
3rd & 2 & 0.3 & 2.5 \\
4th & 0.5 & 0.3 & 2.5\\
5th & 0.5 & 0.1 & 7.6\\
6th & 3 & 0.3 & 2.5 
\end{bmatrix}
\]
By mistake, the sensor was set to 5Hz. Running FFT on the high frequency data does not give sensible results. The raw acceleration data is what has been analysed here. Run number 2 and 6 are equal. 

% bokser på bølgemakern
\begin{figure*}[htbp]
    \centering
    \begin{subfigure}[b]{0.3\textwidth}
        \centering
        \includegraphics[width=\textwidth, angle =-90]{IMG_7113.jpeg}
        \caption{First run}
        \label{fig:subfig100v}
    \end{subfigure}
    \hfill
    \begin{subfigure}[b]{0.3\textwidth}
        \centering
        \includegraphics[width=\textwidth, angle =-90]{IMG_7116.jpeg}
        \caption{Second run, now at an 80 degree angle. This should have been set at a proper 90 degree angle}
        \label{fig:subfig101v}
    \end{subfigure}
    \hfill
    \begin{subfigure}[b]{0.3\textwidth}
        \centering
        \includegraphics[width=\textwidth, angle =-90]{IMG_7169.jpeg}
        \caption{Placement of the IMU and the battery.}
        \label{fig:subfig102v}
    \end{subfigure}
    \caption{IMUs mounted on the wave maker. The box was unnecessary. A bracket to perfectly control the rotation between each run would have be better.}
    \label{fig:three_subfigures}
\end{figure*}




%%%%%%%%%%%%%%% er detta discussion???

%file: small_wave i utklippsmappen.


%see picture: 

%Distance from wave maker to probes: 1st probe: 8m, 2nd probe:10m, 3rd probe (in between 2 and 4, but uncertain), 4th probe: 11.5m. 

%\subsubsection{junk data}

%%%%%%%% accelerometer validation %%%%%%%
\begin{figure}[htbp]
        \centering
        \includegraphics[width=\textwidth]{Log38_run2_wave2_f02f03.jpg}
        \caption{Accelerometer validation. Run 2, wave 2, cutoff frequency f1=0.2, f2=0.3. This figure, and the one below should ideally give identical results. Knowing we have a low sample rate, we should use this as to understand the limitations of our IMU. A direct comparison has not been done.}
        \label{fig:}
\end{figure}

\begin{figure}
        \centering
        \includegraphics[width=\textwidth]{Log38_run2_wave6_f02f03.jpg}
        \caption{Accelerometer validation. Run 2, wave 6, cutoff frequency f1=0.2, f2=0.3.}
        \label{fig:}
\end{figure}



%%%%%%%%%%%%%%%% COMMENT below
\begin{comment}
... dette må nesten bli en oppskrift på hvordan gjøre det bedre neste gang. rekkefølgen er viktig. hva MÅ gjøres først?
1. 
2. Om IMU'en
3. 3D print. plassere dingsen i dead center? god vektfordeling.
4. 
5. wave maker set up
6. computer trigger set up
7. 
8. probe data 
9. data analysis
10. 
11. 
%%%%%
teori fra sist gang:
\subsection{Theory}
Waves 100, 101, 102 are deep water Stokes wave with $ak < 3$. Waves 200, 201,202, are linear.
The dispersion relation for 3rd order Stokes waves:
\begin{align}
\omega = \sqrt{(g\cdot k \cdot (1 + k^2 \cdot A^2))} 
\end{align}
The equation for finding the velocity profile below the crest:
\begin{align}
u = (g\cdot\epsilon/\omega)\cdot\cos(\theta) e^{k\cdot y}
\end{align}

Dispertion relation for linear waves finite depth:
\begin{align}
\omega = \sqrt{(g\cdot k \cdot \tanh(k\cdot h))}
\end{align}
The equation for finding the velocity profile below the crest:
\begin{align}
u = (((g\cdot \epsilon/ \omega)\cdot \cos(\theta))/\cosh(k\cdot h)) \cdot (\cosh(k\cdot (y+h)))
\end{align}


%setup: how did we measure stuff?
%how did we build the thing?
% a test ring was made
%%%%
\end{comment}
