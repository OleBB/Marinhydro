\documentclass{article}

\usepackage{simonspreamble}

\usepackage[toc]{appendix}
\usepackage{xcolor}
\usepackage{cancel} % \cancel, \bcancel,  \xcancel  artig: \cancelto
\usepackage{multicol}
\usepackage{blindtext}
\usepackage{lipsum}  %lorem ipsum
\usepackage{comment}  
\usepackage{placeins} %for å trikse med bildeplassering \FloatBarrier

\definecolor{red}{RGB}{255,0,0}
\definecolor{blue}{RGB}{0,0,255}
\definecolor{green}{RGB}{0,128,0}

\renewcommand\thesubsection{\alph{subsection})} %Gjør at subsection blir a) i stedet for 1.1.1
\newcommand{\vect}[1]{\boldsymbol{{#1}}} %en ny kommando som sier at \vect gir fet skrift. det er alternativ vektornotasjon.

%\begin{multicols*}{2}
%\end{multicols*}

%%%%%% Skriv emnekode her %%%%%%
\makeatletter
\edef\headcourse{MEK4420} %subtitle name and course header
\makeatother

%%%%%% tittel, namn, dato %%%%%%
\title{student task}
\subtitle{\textit{\headcourse}}
\author{\textsc{Ole Sandok}}
\date{March 2025} %\textsuperscript{17}

%%%%%% header title %%%%%%
\makeatletter
\edef\headtitle{\@title}
\makeatother

\begin{document}
\maketitle 
\thispagestyle{fancy}

\section{Introduction}

\section{7.1}
%\subfile{SECTIONS/710BVP/bvp.tex}

\section{7. 2}
%\subfile{SECTIONS/720BVPGREEN/bvpgreen.tex}

\section{7. 3}
%\subfile{SECTIONS/730_40INTEGRAL/integral.tex}

\section{7. 5. 1}
%\subfile{SECTIONS/751LAGBOKS/lagboks.tex}

\section{7.5.2 - Løs integrallikningen med kjente variabler}
%\subfile{SECTIONS/752SOLVE/solve.tex}

\section{7.5.3 - Solution of the heave problem}
%\subfile{SECTIONS/753SOLUTION/753solution.tex}

\section{7.6 - Far field behaviour of phi 2}
%\subfile{SECTIONS/760FARFIELD/farfield.tex}
\textcolor{red}{arbeid gjenstår: løse. rename 143 til min fra section 730-40. Taste inn (103)? og vise tydeligere overgang}

\section{7.7 Utgående bølgeamplitude}
%\subfile{SECTIONS/770AMPLITUDENUTOVER/amplitudenutover.tex}
\textcolor{red}{se over hva i denne 7.7 oppgaven som samsvarer med neste 7.8}

\section{7.8 Added mass}
%\subfile{SECTIONS/780ADDEDMASS/addedmass.tex}
\textcolor{red}{fikse addert masse for alle 4 boksene. fikse b22}

\section{7.10 Diffraksjonsproblemet}
%\subfile{SECTIONS/7101DIFFRAKSJON/7101diffraksjon.tex}

\section{7.10.1 Exciting force}
\subfile{SECTIONS/7101EXCITING/7101exciting.tex}
\textcolor{red}{task: obtain numerically the excinting force. }

\section{7.10.2 Haskind relations}
\subfile{SECTIONS/7102HASKIND/7102haskind.tex}
\textcolor{red}{fortsett å skrive av og så er oppgaven å VISE AT X2/rhog = i A2-infty. Så ber oppgaven om å KALKULERE eksitasjonskraften X2/rhog  for de 3/4 seksjonene ved bruk av 155, 158 og v2 162}

\section{Kladd}
%\subfile{SECTIONS/KLADD/kladd.tex}

\section{References}
[1]: Open Met Buoy, J. Rabault - DOI: 10.13140/RG.2.2.15826.07368



\end{document}

