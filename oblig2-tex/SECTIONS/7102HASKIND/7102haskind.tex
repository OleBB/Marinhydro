Haskind-relasjonene finner vi fra følgende 
\begin{equation}
\frac{X_2}{\mathrm{i} \omega \rho}  =  - \int_{S_B}  ( \phi_0 n_2 + \phi_7 n_2) dS = - \int_{S_B}  \Big( \phi_0 \frac{\partial \phi_2}{\partial n} + \phi_7 \frac{\partial \phi_2}{\partial n} \Big),
\end{equation}
der grenseverdibetingelsen $\frac{\partial \phi_2}{\partial n} = n_2$ er brukt. Videre har vi at potensialet $\phi_2$ og $\phi_7$ tilfredstiller den samme grenseverdibetingelsen ved: a) den frie overflaten: 
... etc ..

dette gir at 
\begin{equation}
\int_{S_B}  \big( \phi_7 \frac{\partial \phi_2}{\partial n} -\phi_0 \frac{\partial \phi_2}{\partial n}  \big) = 0. 
\end{equation}
...
%hopper
%% ned til 
%
...
Første haskind-relasjon
\begin{equation}
\frac{X_2^{\text{Haskind v1}}}{i \omega \rho} = -\int_{S_B}  \big( \phi_0 \frac{\partial \phi_2}{\partial n} -\phi_2 \frac{\partial \phi_0}{\partial n}  \big) dS 
\end{equation}

Andre haskind-relasjon
\begin{equation}
\frac{X_2^{\text{Haskind v2}}}{i \omega \rho} = \int_{S_\infty}  \big( \phi_0 \frac{\partial \phi_2}{\partial n} -\phi_2 \frac{\partial \phi_0}{\partial n}  \big) dS 
\end{equation}

%Feltet langt unna, til $\phi_2$ er gitt ved $A$


