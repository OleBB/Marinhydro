I diffraksjonsproblemet holdes geometrien fast. Fluidets bevegelse er gitt ved hastighetspotensialet

\begin{equation}
\Phi_D(x,y,t) = Re\Big(A  \phi_D(x,y) e^{\mathrm{i} \omega t} \Big), 
\end{equation}
der $A$ er amplituden og $\phi_0$ potensialet til innkommende bølger. Potensialet D finner vi fra $\phi_D(x,y) =  \phi_0(x,y) +  \phi_7(x,y)$. 
$\phi_0(x,y) = \frac{\mathrm{i} g}{\omega}e^{Ky -\mathrm{i} K x}$. Og $K = \frac{\omega^2}{g}$. Spredningen $\phi_7$ er ukjent.

Integrallikningen vi bruker for å bestemme summen $\phi_D =  \phi_0 +  \phi_7$ til et punkt $(\bar{x},\bar{y})$ på $S_B$ er
\begin{equation}
    -\pi \phi_D(\bar{x},\bar{y})  + \int_{S_B}   \phi_D  \frac{\partial G }{\partial n}dS = -2\pi \phi_0(\bar{x},\bar{y}) 
\end{equation}


